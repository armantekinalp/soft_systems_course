% Created 2019-04-26 Fri 09:43
% Intended LaTeX compiler: pdflatex
\documentclass[11pt]{article}
\usepackage[utf8]{inputenc}
\usepackage[T1]{fontenc}
\usepackage{graphicx}
\usepackage{grffile}
\usepackage{longtable}
\usepackage{wrapfig}
\usepackage{rotating}
\usepackage[normalem]{ulem}
\usepackage{amsmath}
\usepackage{textcomp}
\usepackage{amssymb}
\usepackage{capt-of}
\usepackage{hyperref}
\usepackage{awesomebox}
\usepackage{booktabs}
\usepackage{placeins}
\usepackage{siunitx}
\usepackage{minted}
\usepackage{cleveref}
\author{Mattia Gazzola}
\date{}
\title{PovRay : Setup and Tutorials\\\medskip
\large ME498: Computational modeling and optimization}
\hypersetup{
 pdfauthor={Mattia Gazzola},
 pdftitle={PovRay : Setup and Tutorials},
 pdfkeywords={},
 pdfsubject={},
 pdfcreator={Emacs 27.0.50 (Org mode 9.2)},
 pdflang={English}}
\begin{document}

\maketitle
\textbf{Issue Date} : \today

\textbf{Teaching Assistant} : Tejaswin Parthasarathy, \texttt{tp5@illinois.edu}

\section{Setup}
\label{sec:orged45d52}
Installing \texttt{PovRay} is straightforward in a *nix system. Some ways are:

\begin{itemize}
\item If you are running \texttt{Ubuntu}, you can install the \texttt{PovRay} executable and
libraries using the \texttt{apt} module, like so:
\begin{minted}[]{sh}
# fetches up-to-date package information from
  #the ubuntu aptitude repository
sudo apt-get update

# upgrades all packages
sudo apt-get upgrade

# install povray, if not already there
sudo apt-get install povray
\end{minted}
Beware that you may run into this issue:

\texttt{povray: cannot open the user configuration file /home/.povray/3.7/povray.conf: No such file or directory}

Running this command in your home directory (as borrowed from \href{https://askubuntu.com/questions/628496/using-pov-ray-installed-via-the-apt}{this SO thread}
 ) seems to resolve this issue.

\texttt{ln -s /etc/povray/ .povray}

\item If you are running \texttt{Ubuntu}, you can also install the \texttt{PovRay} executable and
libraries using the directions provided in \href{http://www.povray.org/download/linux.php}{this official installation link}.
\item Alternatively you can install compile from source following the step-by-step
instructions posted \href{https://askubuntu.com/a/414192}{in this Stack Overflow answer}.
\end{itemize}

If you are running Windows, you can

\begin{itemize}
\item Use the Linux subsystem that you setup at the beginning of this semester to
invoke the commands listed for \texttt{Ubuntu} users above.
\item Directly install the Povray binary from \href{http://www.povray.org/download/}{here}. I do not recommend this option
as the scripts that I have uploaded requires some variant of \texttt{sh}, not found
in a Windows environment.

\item If you are running other distros (\texttt{Debian} etc.), invoke the appropriate package
manager, with the same commands above.
\item Alternatively, if you are on \texttt{MacOS} or \texttt{Linux} based systems and are using
\texttt{Homebrew/Linuxbrew}, then install \texttt{PovRay} using:
\begin{minted}[]{sh}
# install povray, if not already there
brew install povray
\end{minted}
Note that \texttt{MacOS} users can also compile code from source (similar to \texttt{Ubuntu} users)
\end{itemize}

\section{Tutorials}
\label{sec:org0e98fd4}
Here are some tutorials for getting familiar with \texttt{PovRay}. As mentioned in
class, I will be giving you some scripts to do the rendering of the snake from
your \texttt{Python} implementation:
\begin{itemize}
\item Nice gentle introduction : \url{http://theringlord.org/derakoninstructions/povtute.html}
\item This video series :
\end{itemize}
\url{https://www.youtube.com/playlist?list=PLlW5kivBxMYtCFiSnhlhceFsTSCeuJImm}
\begin{itemize}
\item The basic tutorials from the official PovRay site: \url{https://www.povray.org/documentation/3.7.0/t2\_0.html}
\item This site is the Godfather of all sites PovRay:
\url{http://f-lohmueller.de/pov\_tut/pov\_\_eng.htm} and has many tricks, tips and tutorials
\end{itemize}
\section{\texttt{ffmpeg} setup}
\label{sec:org2582b40}
Installing \texttt{ffmpeg} is also straightforward in a *nix system. We will be
requiring \texttt{ffmpeg} to stitch together images created by \texttt{PovRay} to create
animations.

\begin{itemize}
\item If you are running \texttt{Ubuntu}, you should already have it installed. If not
(there is no preinstalled \texttt{ffmpeg} in \texttt{Ubuntu 14.04}),
run	the following commands using the \texttt{apt} module, like so:
\begin{minted}[]{sh}
# Add external repository containing ffmpeg
sudo add-apt-repository ppa:mc3man/trusty-media

# fetches up-to-date package information from
# the ubuntu aptitude repository
sudo apt-get update

# upgrades all packages
sudo apt-get upgrade

# install ffmpeg, if not already there
sudo apt-get install ffmpeg
\end{minted}
\end{itemize}

If you are running Windows, you can

\begin{itemize}
\item Use the Linux subsystem that you setup at the beginning of this semester to
invoke the commands listed for \texttt{Ubuntu} users above.
\item If you are running other distros (\texttt{Debian} etc.), invoke the appropriate package
manager, with the same commands above.
\end{itemize}

If you are running \texttt{MacOS} or \texttt{Linux}, you can use \texttt{Homebrew/Linuxbrew}, then
install \texttt{ffmpeg} using:
\begin{minted}[]{sh}
# install ffmpeg, if not already there
brew install ffmpeg
\end{minted}
\section{\texttt{ffmpeg} commands}
\label{sec:org3428bce}
The following \texttt{ffmpeg} command finds images in a given folder and creates a
nice movie out of them all. It has been added to the source script by default.
\section{Putting them together}
\label{sec:org973d6fb}
The \texttt{dump\_snake.py} script provided can dump data from the rod into a \texttt{PovRay}
recognized format. The typical use case, and the generated image is shown
below:
\begin{minted}[]{python}
from dump_snake import dump_snake_to_povray

# Fake rod objects
# x, y and z of nodes
x_rod = np.linspace(0., 1., 50)
y_rod = 0.1 * np.sin(5 * np.pi * x_vals)
radius = 0.05
z_vals = 0.0 * x_vals - radius

# Put the data as a (npts,3) or (3, npts) object
data_arr = np.vstack((x_vals, y_vals, z_vals))

# Folder where to dump data, optional
prefix = "./data/"

# Finally dump snake data into povray objects
# Look at the function docstring for more details
dump_snake_to_povray(0, data_arr, radius, prefix)
\end{minted}
Note that there are caveats in using this function (which are detailed in its
docstring). The caveats however are not too restrictive, and typically your
snake should render fine.

Using the \texttt{render\_pov.sh} script, you can then ask \texttt{PovRay} to render the
files generated by python as images. You can use it like so:

\begin{minted}[]{bash}
# The argument passed to render_pov is
# where your data dumped from Python is located
bash render_pov.sh ./data/
\end{minted}

Finally a sample image obtained from running the above code is shown below.
Note that the script also generates the \texttt{snake.mp4} animation file which
stitches all such images together.

\begin{figure}[htbp]
\centering
\includegraphics[width=.9\linewidth]{code/test_images/snake_0000000.png}
\caption{\texttt{Povray} rendered snake}
\end{figure}
\end{document}
