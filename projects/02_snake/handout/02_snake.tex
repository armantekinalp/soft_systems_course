% Created 2019-03-04 Mon 22:05
% Intended LaTeX compiler: pdflatex
\documentclass[11pt]{article}
\usepackage[utf8]{inputenc}
\usepackage[T1]{fontenc}
\usepackage{graphicx}
\usepackage{grffile}
\usepackage{longtable}
\usepackage{wrapfig}
\usepackage{rotating}
\usepackage[normalem]{ulem}
\usepackage{amsmath}
\usepackage{textcomp}
\usepackage{amssymb}
\usepackage{capt-of}
\usepackage{hyperref}
\usepackage{awesomebox}
\usepackage{booktabs}
\usepackage{placeins}
\usepackage{siunitx}
\usepackage{minted}
\usepackage{cleveref}
\author{Mattia Gazzola}
\date{}
\title{Project 2: Modeling and optimization of a slithering snake\\\medskip
\large ME498: Computational modeling and optimization}
\hypersetup{
 pdfauthor={Mattia Gazzola},
 pdftitle={Project 2: Modeling and optimization of a slithering snake},
 pdfkeywords={},
 pdfsubject={},
 pdfcreator={Emacs 27.0.50 (Org mode 9.2)},
 pdflang={English}}
\begin{document}

\maketitle
\textbf{Issue Date} : \today

\textbf{Teaching Assistant} : Tejaswin Parthasarathy, \texttt{tp5@illinois.edu}

\textbf{Submission Date/Time}: Wednesday 11:59 PM, 27 March, 2019

\textbf{Submission Instructions}:
\begin{itemize}
\item You need to submit both your presentation (as \texttt{.pdf}, \texttt{.ppt[x]} or \texttt{.key}) and code
(as \texttt{.py} or \texttt{.ipynb}) on Compass, in separate archived files (see below)
\item Submit your code file(s) packaged as a \texttt{.zip} or \texttt{.tar.gz} files (please do not use
\texttt{.7z} or other formats) and name the compressed file as \texttt{<net\_id>\_code\_02}
\item Submit your presentation file(s) packaged as a \texttt{.zip} or \texttt{.tar.gz} files (please do not use
\texttt{.7z} or other formats) and name the compressed file as \texttt{<net\_id>\_slides\_02}
\end{itemize}

\newpage

\section{Guidelines}
\label{sec:orgeea9f5b}
General advice that pertains to solving the dynamics of an idealized
slithering snake:
\begin{enumerate}
\item Validate your code (i.e. test that it captures the physics) before doing
any optimization on your snake model. Some sensible cases for validation are
discussed later on.
\item You will need to couple the physical simulations that you develop in this
project with the optimizer---hence develop code keeping this requirement in mind.
\item Use the CMAes class/function that you developed as part of the previous
project to run the optimization campaign.
\item We will evaluate your work based on your understanding and findings, and
not on the code quality. However since you are learning \texttt{Python}, it is
always a good idea to code efficiently. Use canned algorithms in \texttt{numpy} or
\texttt{scipy} wherever possible.
\item You are free to consult any material you want (including but not limited to
resources in \cref{sec:references}). However you are expected to follow the
Student Code on academic integrity---plagiarism will not be tolerated!
\end{enumerate}

\section{A soft snake}
\label{sec:orgd1f5741}
\subsection{Definition}
\label{sec:orgc48a591}
In \texttt{Python}, implement a code for computing the mechanics of soft filaments
using Cosserat rod theory, following the discussion in class. Construct a
model of a slithering snake using this implementation and furthermore
optimize the gait of the snake for achieving maximal forward speed.

\textbf{BONUS} : Visualize any/all of your results using the \href{http://www.povray.org/}{PovRay} raytracing
library or its \texttt{Python} interface \href{https://github.com/Zulko/vapory}{Vapory}.
\subsection{Details}
\label{sec:org39b5084}
\subsubsection{Governing equations}
\label{sec:org4329c6e}
\begin{enumerate}
\item Numerical discretization
\label{sec:org0f324f0}
Pick the same.
\end{enumerate}
\subsubsection{Simplifications to the governing equations}
\label{sec:org733c01f}
\begin{enumerate}
\item First simplification is done by throwing away useless terms.
\label{sec:org9599b3f}
\item Set e = 1
\label{sec:org41cf07c}
\end{enumerate}
\subsubsection{Validation}
\label{sec:orgc3f792b}
Timoshenko Beam

The parameters for this particular problem are \ldots{}
\subsubsection{Modeling muscular activity and friction}
\label{sec:org229cf45}
\begin{itemize}
\item Parametrization of muscles
\item Including Friction as external force.
\end{itemize}
\subsubsection{Gait optimization for maximal forward speed}
\label{sec:orgebfa432}

More details about the 0/1 Knapsack problem are contained in the lecture slides.
Alternatively, the \href{https://en.wikipedia.org/wiki/Knapsack\_problem}{Wikipedia} page also has a lot of information about this
problem. Quoting the introductory text,
\begin{quote}
`` The knapsack problem or rucksack problem is a problem in combinatorial
optimization: Given a set of items, each with a weight and a value, determine
the number of each item to include in a collection so that the total weight is
less than or equal to a given limit and the total value is as large as possible.
It derives its name from the problem faced by someone who is constrained by a
fixed-size knapsack and must fill it with the most valuable items. ''
\end{quote}

Your task is to find an optimal distribution of items to carry, that
maximizes the overall value while adhering to the weight limit, using both
\begin{itemize}
\item GA
\item CMA
\end{itemize}
and compare the performance of these algorithms on this function. Details about
the dataset for this problem is given below in \cref{sec:data}. Note that this is a constrained
optimization problem and your design choices in the
algorithm need to reflect this. As this seems like a problem with integer
solutions, consider how you can change CMA to reflect this property. As we are
interested in black-box optimization algorithms, we assume we do not know
anything about the problem structure. This means that using a randomly generated
population around \(\mathbf{x}_0 =\mathbf{0}\)---the vector of zeros---to
initialize the algorithm is a good idea.

Finally, after answering all the questions above and implementing your
algorithm, explore the design choices (be it parameters or operators) on the
performance of the algorithms.
\subsection{Settings}
\label{sec:data}
There are two data sets--A, B. Each data set will be posted on Compass as
simple \texttt{numpy npz} files with the knapsack capacity (total weight of the
knapsack), number of different items (which are all assumed to be distinct in
the 0/1 knapsack problem so you \textbf{do not} need to worry about \emph{choosing} same
items), and their values + weights. For convenience you can read the
information using \texttt{numpy}'s \texttt{load} function, shown in the snippet below:

\section{The following resouces may prove useful:}
\label{sec:references}
\begin{itemize}
\item Paper describing the governing equations, numerical algorithm and optimization
of a slithering snake, found \href{https://royalsocietypublishing.org/doi/full/10.1098/rsos.171628}{in this link}.
\item The CMA-ES tutorial @ Arxiv, found \href{https://arxiv.org/pdf/1604.00772.pdf}{here}
\item More information on timestepping schemes found \href{https://cg.informatik.uni-freiburg.de/course\_notes/sim\_02\_particles.pdf}{at this link}
\end{itemize}
\end{document}
